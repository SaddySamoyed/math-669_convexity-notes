\documentclass[lang=cn,11pt]{elegantbook}
\usepackage[utf8]{inputenc}
\usepackage[UTF8]{ctex}
\usepackage{amsmath}%
\usepackage{amssymb}%
\usepackage{graphicx}

\title{Math 669: Topics in con}

\begin{document}
\frontmatter
\tableofcontents
\mainmatter

\chapter{Introduction}

\noindent For vector spaces in this note, it is over $\bR$, with the standard inner product and Euclidean norm when mentioned.

\begin{definition}{interval}
    Given $x,y \in \bR^n$, we define the interval between $x$ and $y$ as the set 
    $$
    [x,y] := \{ \alpha x + (1-\alpha)y \mid 0 \leq \alpha \leq 1  \}
    $$
\end{definition}

\begin{definition}{convex set}
    We say $A \sub V$ is convex if for all $x,y \in A$, we have $[x,y] \sub A$
\end{definition}
\pic[0.3]{assets/lec1(1).png}

There are of some trivial convex sets we can easily think about, like cubes and balls. There are also non-trivial convex sets. Below is an example.
\begin{example}
\begin{theorem}{Schor-Horn Theorem (I.Schor, 1875-1941, A.Horn, 1918-2001)}
\label{Schor-Horn Theorem}
Fix $\lambda_1, \cdots, \lambda_n \in \bR$, consider 
$$
E := \{ A \in \Sym(n) \mid A \text{ has } \{\lambda_i\}_{i=1}^n \text{ as eigenvalues} \} \sub \Sym(n)
$$
Claim:
$$
\{ ((A)_{11}, \cdots, (A)_{nn}) \mid A \in E   \}
$$
, the set of diagonals of all matrices in $E$, is convex. And furthermore, it is \textbf{the smallest convex set containing the vectors} $\{ ( \lambda_{\sigma(1)}, \cdots,  \lambda_{\sigma(n))}) \mid \sigma \in S_n  \}$
\end{theorem}
\noindent Here we take an example for $n=3$. Note $|S_3| = 3! = 6$. The six points formed by permuting $\lambda_1, \lambda_2, \lambda_3$ form a regular hexagon on a hyperplane, and area between is the set of diagoanls of all matrices in $E$.
\pic[0.3]{assets/lec1(2).png}
\begin{proof}
    Later on.
\end{proof}
\end{example}

\begin{example}
\begin{theorem}{Brickman's Thm, 1961, L.Brickman}\label{Brickman's Thm}
Let $q_1, q_2: \bR^n \rar \bR$ be quadratic forms.\\
Define $\phi: S_{n-1} \rar \bR^2$ sending $x \mapsto (q_1(x), q_2(x))$.\\
Claim: if $n \geq 3$, then $\phi(S^{n-1})$ is convex in $\bR^2$.
\end{theorem}


\end{example}

\end{document}